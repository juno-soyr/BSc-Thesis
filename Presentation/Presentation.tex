\documentclass{beamer}
%Information to be included in the title page:
\usetheme{Berlin}
\title{Invariance in the Lambda Calculus through Explicit Substitutions}
\author{Haileselassie Gaspar}
\institute{Vrije Universiteit Amsterdam}
\date{14-06-2025}

\begin{document}

\frame{\titlepage}

\begin{frame}
  \frametitle{Contents}
  \begin{enumerate}
    \item Introduction and Background
    \item Lambda Calculus
    \item The Size-Explosion Problem
    \item Linear Substitution Calculus
    \item Examples
    \item Conclusion
    \item Bibliography
  \end{enumerate}
\end{frame}
\section{Introduction and Background}
\begin{frame}
  \frametitle{What is computation?}
  A property of problems or functions to be solved by some mechanical process. The main method of studying said mechanical processess is done through a \textit{machine model}.
\end{frame}
\begin{frame}
  \frametitle{Turing Machines, $\lambda$-calculus and Invariance}
Models of computation developed at the start of the 20th century to solve -Hilbert's problems-.
  \begin{itemize}
    \item Turing Machines, Alan Turing (1936) -cite-
          \item $\lambda$-calculus, Alonzo Church (1930-1940) -cite-
  \end{itemize}
  These two models, and a third that we will not discuss here, represent the same group of functions. This came to define the idea of computation.
\end{frame}
\begin{frame}
  \frametitle{Equivalence and Invariance of Models}

\end{frame}
\end{document}
%Gotta add bibliography
