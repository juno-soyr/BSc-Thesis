\documentclass{beamer}
\setbeameroption{show notes on second screen=right} % Both
%Information to be included in the title page:
\usepackage{amssymb}
\usetheme{Berlin}
\title{Invariance in the Lambda Calculus through Explicit Substitutions}
\author{Haileselassie Gaspar}
\institute{Vrije Universiteit Amsterdam}
\date{14-06-2025}

\begin{document}

\frame{\titlepage}

\begin{frame}
  \frametitle{Contents}
  \begin{enumerate}
    \item Introduction and Background
    \item The Size-Explosion Problem
    \item Linear Substitution Calculus
    \item High-Level Implementation Systems
    \item Examples
  \end{enumerate}
\end{frame}
\section{Introduction and Background}
\begin{frame}
  \frametitle{Equivalence and Invariance of Models}
  \begin{itemize}
    \item The Church-Turing Thesis - Turing, Kleene, Church, Rosser
    \item The Invariance Thesis - Van Embde Boas
  \end{itemize}
  \note{Turing machines are the foundational measure of computational complexity so when we talk about equivalence and invariance we refer to it WITH RESPECT TO TURING MACHINES.
  We will study INVARIANCE through a cost model. We will only talk about time invariance in this presentation.}
\end{frame}
\section{Size-explosion Problem}
\begin{frame}
  \frametitle{Size exploding Family of $\lambda$-terms}
  \begin{equation}
    \begin{split}
      & t_{0} \equiv yxx \\
      & t_{n+1} \equiv (\lambda x.t_{n})(yxx)
    \end{split}
  \end{equation}
  We will refer to the normal form of this terms by $r_{n}$.
  By induction we can see that $t_{n} \xrightarrow[1]{\beta} (\lambda x .t_{n - 2})y(yxx)(yxx) \xrightarrow[n-1]{\beta} r_{n}$, and that $|r_{n}| \in O(2^{n})$.
  \note{When we talk about the size explosion we refer to it in terms of how a Turing machine would represent this term, which obviously takes $2^n$ steps since space complexity is a lower bound for time complexity on Turing machines}
\end{frame}

\section{Linear Substitution Calculus}
\begin{frame}
  \frametitle{Syntax and operational semantics}
\begin{itemize}
  \item Syntax
    \begin{equation}
    \begin{split}
      & t, u ::= \ x \ | \ \lambda_{lsc}x.t \ | \ tu \ | \ t[x \leftarrow u ] \\
      & S :: = \ \langle \cdot \rangle \ | \ \lambda_{lsc}x.S \ | \ St \ | \ tS \ | \ S[x \leftarrow t] \\
      & L ::= \ \langle \cdot \rangle \ | \ L[x \leftarrow t]
    \end{split}
  \end{equation}

  \item Operational Semantics
        \begin{equation}
          \begin{split}
            & L \langle \lambda_{lsc} x.t \rangle u \rightarrow_{dB} L \langle t [x \leftarrow u] \rangle \\
            & S \langle x \rangle [x \leftarrow u] \rightarrow_{ls} S \langle u \rangle [x \leftarrow u]
          \end{split}
        \end{equation}
\end{itemize}
\end{frame}
\begin{frame}
  \frametitle{Unfolding of Shared terms}
  We introduce the operation $\downarrow$ in order to convert $\lambda_{LSC}$-terms to regular $\lambda$-terms.
  \begin{equation}
  t[x \leftarrow u]\downarrow = t\downarrow \{ x \leftarrow u \downarrow \}
\end{equation}
And the contextual unfolding of a term:
\begin{equation}
  t \downarrow_{S[x \rightarrow u]} = t \downarrow_{S} \{ x \rightarrow u \downarrow \}
\end{equation}
\end{frame}
\section{High-level Implementation Systems}
\begin{frame}
  Let $\rightsquigarrow$ be a deterministic $\lambda$-strategy and $\rightsquigarrow_{X}$ be a strategy on the LSC. The pair $(\rightsquigarrow , \rightsquigarrow_{X})$ is called a high-level implementation system if it has the following properties:
  \begin{enumerate}
    \item Normal Form Equality
    \item Projection
    \item Trace
    \item Syntactic Bound
  \end{enumerate}
\end{frame}
\begin{frame}
  \frametitle{Useful Derivations}
  An applicative context is defined by $A = S\langle Lt \rangle$. \\
  A useful step is either a dB-step or a ls-step $S \langle x \rangle \rightarrow S \langle r \rangle$ so that the unfolding $r \downarrow_{S}$:
  \begin{enumerate}
          \item Either ontains a $\beta$-redex
          \item Or is an abstraction and S is an applicative context.
  \end{enumerate}
\end{frame}
\section{Examples}
\begin{frame}
  \frametitle{Size exploding Terms revisited}
  Taking $u = yxx$ for readability:
  \begin{equation}
    \begin{split}
  & t_{2} \equiv (\lambda x.(\lambda x . (yxx))(yxx))(yxx) \xrightarrow[2]{\beta} y(yuu)(yuu) \equiv r_{2} \\
  & t_{2} \xrightarrow[2]{dB} (yxx)[x \leftarrow yxx][x \leftarrow yxx]
    \end{split}
  \end{equation}
\end{frame}
\end{document}
%Gotta add bibliography
