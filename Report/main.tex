\documentclass[12pt]{article}
\title{Thesis}
\author{Haileselassie Gaspar}

\begin{document}
\maketitle

\section{Abstract}
\section{Introduction}
\section{Theoretical Basis}
As stated before the measure employed to analyze the time invariance of lambda calculus, or, said differently, its universality, is the number of transitions in a turing machine. If the implementation introduced in \textbf{Beta reduction invariance Paper citation --} is correct, then by means of the Linear Substitution Calculus, it is possible to represent even size-exploding terms in Turing machines. The proof will be dividied into two sections, and this paper will focus on the implementation of the first in Haskell.
\subsection{High-Level Implementation}



\end{document}
